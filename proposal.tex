\documentclass[dvips,12pt]{article}


\setlength{\oddsidemargin}{0.0in}
\setlength{\textwidth}{6.5in}
\setlength{\topmargin}{-1in}
\setlength{\textheight}{10.5in}


\begin{document}


\title{ECSE 548: Introduction to VLSI Systems\protect\\Project Proposal - Group 1}

\author{Xinchi Chen, Kaushik Boga, Georgi Kostadinov, Mojing Liu}
\date{\today}

% You can leave out "date" and it will be added automatically for today
% You can change the "\today" date to any text you like


\maketitle

% This command causes the title to be created in the document

\section{Project Definition}

% An article style is separated into sections and subsections with 
%   markup such as this.  Use \section*{Principles} for unnumbered sections.

Traditional error mitigation techniques such as Error Correcting Code (ECC)
provide limited error coverage rate at the cost of performance and area overhead. Another solution widely used in
industry, namely Dual Modular Redundancy (DMR) in Lockstep, consists of two redundant cores executing the same application
in lockstep. The results of both cores are compared and the data gets commited to the outside world only when the two results match. 
Resolving the issue of performance impact, lockstep execution has been shown to be a great waste of computational
power. Therefore, in this project we propose to implement Execution fingerprinting which will greatly enhance the performance
of a DMR system.


\section{Targets}

During the course of this project, the plan is to:
\begin{itemize}
\item Design a single core execution fingerprinting system that uses CRC for fingerprinting
\item Design 2 CRC layouts to fit the MIPS8 multicycle processor to generate the fingerprints
\item Store the generated fingerprints in a buffer
\item Evaluate the performance impact, area overhead and error coverage of the system
\end{itemize}

The fingerprinting module will monitor the output ports of the current MIPS8 processor. 
The output will be compressed using the CRC circuit, generating a fingerprint. This fingerprint will then be
stored in a buffer.

By the mid-project status report, we target to finish the layout of both CRC circuits for a functional
fingerprinting circuit. From that point, we will be able to optimize the circuit according to the behavior
of the MIPS8 core. Then, a series of benchmarks will be executed in order to analyse the performance
impact, area overhead and error coverage of the circuit.



\end{document}
